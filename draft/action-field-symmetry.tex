\documentclass{article}
\usepackage{amsmath}
\usepackage{physics}
\usepackage{enumitem}
\usepackage{txfonts}
\usepackage[a4paper, top=1cm, bottom=2cm, left=2cm, right=2cm, includehead, includefoot]{geometry}

\title{Fundamental Symmetries in Classical Field Theories}
\author{Aaron W. Tarajos}

\begin{document}

\maketitle

\section{Theorem}
We will formally show that invariance of action$\implies$covariance of the field equations by three proofs including; a finite proof for discrete and continuous quasi-symmetry, an infinitesimal proof for a continouous quasi-symmetry from the infinitesimal transformation, and a proof of the infinitesimal tranformation for $x$-locally without the use of functional derivatives. Theorem;
If a local action functional $S_V\left[\phi\right]$ has a quasi-symmetry transformation
\begin{equation}
	\phi^\alpha (x) \to \phi^{\prime\alpha}(x^\prime), \quad x^\mu \to x^{\prime\mu},
\end{equation}
then the equations of motion
\begin{equation}
	e_{\alpha}(\phi(x),\partial\phi(x),\ldots ; x)~:=~\frac{\delta S_V[\phi]}{\delta \phi^{\alpha}(x)}~\approx~0
\end{equation}
must have a symmetry with respect to the same transformation;
\begin{equation}
	e_{\alpha}(\phi^{\prime}(x^{\prime}),\partial^{\prime}\phi^{\prime}(x^{\prime}),\ldots ; x^{\prime})~\approx~e_{\alpha}(\phi(x),\partial\phi(x),\ldots ; x).
\end{equation}

\subsection{Invariance and Covariance}
An object is said to be invariant if under some transformation the object remains unchanged. That is for an arbitrary transformation of a field $\phi$;
\[
	\phi^\prime  = \phi
\]
Similarly, an object is covariant if its form is preserved when the fields and coordinates are transformed.
\[
	F\left[\phi^\prime\right] = F\left[\phi\right]
\]
Intuitively it may seem that a system that is covariant are also invariant but we find that is not the case. This is beyond the scope of this paper but one example of an action that is covariant but not invariant is the weak interaction.

\subsection{Quasi-symmetry}
We define an action functional $S_V\left[\phi\right]$ as the integral of the $n$-form Lagrangian $\mathbb{L}$ over a region of spacetime $V$;
\begin{equation}
	S_V\left[ \phi \right] \coloneqq \int_V \mathbb{L}, \quad \mathbb{L} \coloneqq \mathcal{L}\ d^nx\ .
\end{equation}
The Lagrangian is defined by the integral of the Lagrangian density $\mathcal{L}$ in $n$-dimensional space. Then, the action functional $S_V\left[\phi\right]$ has a quasi-symmetry if it changes by a boundary integral such that the transformed action functional is equal to the original action functional plus the same boundary integral over the transformed spacetime region $V^\prime$:
\begin{equation}
	S_V^\prime \left[\phi^\prime\right] + \int_{\partial V^\prime} d^{n-1} (\dots) = S_V \left[\phi\right] + \int_{\partial V} d^{n-1} (\dots)
\end{equation}

\section{Equations of Motion}
We have defined the action functional to be invariant under infinitesimal variations in the field $\delta \phi^\alpha(x)$ and therefore;
\[
	\frac{\delta S_V[\phi]}{\delta \phi^{\alpha}(x)} = 0 \ ,
\]
giving us the equations of motion;

\begin{equation}
	e_{\alpha}(\phi(x),\partial\phi(x),\ldots ; x) \coloneqq \frac{\delta S_V[\phi]}{\delta \phi^{\alpha}(x)} \approx 0 \ .
\end{equation}

\section{Formal Finite Proof}
Starting with the equations of motion;

\[
	e_{\alpha}(\phi(x),\partial\phi(x),\ldots ; x) = \frac{\delta S_V[\phi]}{\delta \phi^{\alpha}(x)}
\]
by (5) we can say that

\[
	\delta S_{V^\prime} \left[\phi^\prime\right] = \delta S_{V^\prime} \left[\phi\right]
\]
and therefore

\[
	e_{\alpha}(\phi(x),\partial\phi(x),\ldots ; x) = \frac{\delta S_{V^\prime} \left[\phi^\prime\right]}{\delta \phi^{\alpha}(x)}
\]
by the chain rule;

\begin{align*}
	\frac{\delta S_{V^\prime} \left[\phi^\prime\right]}{\delta \phi^{\alpha}(x)} &= \int_{V^\prime}d^nx^\prime \frac{\delta S_{V^\prime} \left[\phi^\prime\right]}{\delta \phi^{\prime \alpha}(x^\prime)} \frac{\delta \phi^{\prime \alpha}(x^\prime)}{\delta \phi^{\alpha}(x)} \\
	\frac{\delta S_{V^\prime} \left[\phi^\prime\right]}{\delta \phi^{\alpha}(x)} &= \int_{V^\prime}d^nx^\prime e_\alpha\left(\phi^\prime(x^\prime),\partial^\prime \phi^\prime(x^\prime), \dots ; x^\prime \right) \frac{\delta \phi^{\prime \alpha}(x^\prime)}{\delta \phi^{\alpha}(x)} \\
	e_{\alpha}(\phi(x),\partial\phi(x),\ldots ; x) &= \int_{V^\prime}d^nx^\prime e_\alpha\left(\phi^\prime(x^\prime),\partial^\prime \phi^\prime(x^\prime), \dots ; x^\prime \right) \frac{\delta \phi^{\prime \alpha}(x^\prime)}{\delta \phi^{\alpha}(x)} \\
	e_{\alpha}(\phi(x),\partial\phi(x),\ldots ; x) &= e_\alpha\left(\phi^\prime(x^\prime),\partial^\prime \phi^\prime(x^\prime), \dots ; x^\prime \right) \quad \blacksquare
\end{align*}





\end{document}

