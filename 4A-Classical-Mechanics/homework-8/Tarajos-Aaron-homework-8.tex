\documentclass{article}
\usepackage{graphicx}
\usepackage{amsmath}
\usepackage{pgfplots}
\usepackage{physics}
\usepackage{cancel}
\usepackage{enumitem}
\usepackage{txfonts}

\pgfplotsset{compat=1.18}

\usepackage[a4paper, top=1cm, bottom=2cm, left=2cm, right=2cm, includehead, includefoot]{geometry}

\begin{document}

\noindent
Physics 4A - Classical Mechanics \hfill Prof. Roger King

\noindent\rule{\textwidth}{0.4pt}

\begin{center}
    \textbf{\LARGE Homework 8} \\
    \vspace{12pt}
    \large Aaron W. Tarajos \\
    \textit{\today}
\end{center}

\noindent\rule{\textwidth}{0.4pt}

\section*{Problem 1}
A 0.315-kg particle moves from an initial position $\va{r}_1 = 2.00\ \vu{i} - 1.00\ \vu{j} + 3.00\ \vu{k}$ m to a final position $\va{r}_2 = 4.00\ \vu{i} - 3.00\ \vu{j} - 1.00\ \vu{k}$ m while a force $\va{F} = 2.00\ \vu{i} - 3.00\ \vu{j} + 1.00\ \vu{k}$ N acts on it. What is the work done by the force on the particle?

\subsection*{Solution}
The distance traveled by the particle, $\va{d}$, is equal to the difference in final and initial position
\begin{align*}
	\va{d} &= \va{r_2} - \va{r_1} \\
	       &= \left( 4.00 - 2.00 \right)\ \vu{i} + \left( -3.00 + 1.00 \right)\ \vu{j} + \left( -1.00 -3.00 \right)\ \vu{k} \\
	       &= 2.00\ \vu{i} - 2.00\ \vu{j} - 4.00\ \vu{k}
\end{align*}
Then work is the dot product of $\va{F}$ and $\va{d}$
\[
	W = \va{F} \cdot \va{d} = (2)(2) + (-2)(-3) + (-4)(1) = \boxed{6\ \text{J}}
\]

\section*{Problem 2}
Compute the kinetic energy for each of the cases below. Through what distance would a 800-N force have
to act to stop each object? \\
(a) A 150-g baseball moving at 40 m/s; \\
(b) a 13-g bullet from a rifle moving at 635 m/s; \\
(c) a 1500-kg Corvette moving at 250 km/h; \\
(d) a $1.8 \times 10^5$-kg Concorde airliner moving at 2240
km/h.

\subsection*{Solution}
The kinetic energy is given by

\begin{equation}
	k = \frac{1}{2}mv^2
\end{equation}
Then

\[
	k = Fd \implies d = \frac{k}{F}
\]
Using these equations to solve for each part;
\subsubsection*{Part a:}
\[
	k = \frac{1}{2} (150)(40)^2 = \boxed{120\ 000\ \text{J}}
\]
and

\[
	d = \frac{120\ 000}{800} = \boxed{150\ \text{m}}
\]

\subsubsection*{Part b:}
\[
	k = \frac{1}{2} (13)(635)^2 = \boxed{2.62 \times 10^6\ \text{J}}
\]
and

\[
	d = \frac{2.62 \times 10^6}{800} = \boxed{3276.20\ \text{m}}
\]

\subsubsection*{Part c:}
\[
	k = \frac{1}{2} (1500)(250)^2 = \boxed{46\ 875\ 000\ \text{J}}
\]
and

\[
	d = \frac{120\ 000}{800} = \boxed{150\ \text{m}}
\]

\section*{Problem 3}
Compute the kinetic energies for each of the following. What force would be required to stop each object in
1.00 km? \\
(a) The $8.00 \times 10^7$-kg carrier Nimitz moving at 55 km/h; \\
(b) a $3.4 \times 10^5$-kg Boeing 747 moving at 1000 km/h; \\
(c) the 270-kg Pioneer 10 spacecraft moving at 51,800 km/h.

\section*{Problem 4}
A 1.50-kg block is moved at constant speed in a vertical plane from position 1 to position 3 via several
routes shown in the figure. Compute the work done by gravity on the block for each segment indicated,
where $W_{ab}$ means work done from a to b. \\
(a) $W_{13}$ \\
(b) $W_{12} + W_{23}$ \\
(c) $W_{14} + W_{43}$ \\
(d) $W_{14} + W_{45} + W_{53}$

\section*{Problem 5}
What is the work needed to lift 14.7 kg of water from a well 11.0 m deep. Assume the water has a constant
upward acceleration of 0.700 m/s$^2$.

\section*{Problem 6}
The variation of a force with position is shown in the figure below. Find the work from
(a) $x = 0$ to $x = -A$ \\
(b) $x = +A$ to $x = 0$

\section*{Problem 7}
Consider a particle on which several forces act, one of which is known to be constant in time:
$\va{F}_1 = 3.00\ \vu{i} + 4.00\ \vu{j}$ N. As a result, the particle moves along a straight path from a Cartesian coordinate of
(0.00 m, 0.00 m) to (5.00 m, 6.00 m). What is the work done by $\va{F}_1$?

\section*{Problem 8}
A bungee cord exerts a nonlinear elastic force of magnitude $F(x) = k_1 x + k_2 x^3$, where $x$ is the distance the cord is stretched, $k_1 = 204$ N/m and $k_2 = -0.233$ N/m$^3$. How much work must be done on the cord to stretch it 16.7 m?

\end{document}
