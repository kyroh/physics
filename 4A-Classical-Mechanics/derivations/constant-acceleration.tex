\documentclass{article}
\usepackage{graphicx}
\usepackage{amsmath}
\usepackage{pgfplots}
\usepackage{siunitx}
\usepackage{cancel}

\newcommand{\mile}{\text{mi}}
\newcommand{\gallon}{\text{gal}}
\newcommand{\kilo}{\text{k}}
\newcommand{\liter}{\text{L}}
\newcommand{\meter}{\text{m}}
\newcommand{\second}{\text{s}}
\newcommand{\foot}{\text{ft}}

\pgfplotsset{compat=1.18}

\usepackage[a4paper, top=1cm, bottom=2cm, left=2cm, right=2cm, includehead, includefoot]{geometry}

\begin{document}
\flushleft
Physics 4A - Classical Mechanics \hfill Prof. Roger King\\
\hrule

\begin{center}
    \vspace{14pt}
    \textbf{\LARGE Derivation of Equations for Constant Acceleration} \\
    \vspace{12pt}
    \text{\large Aaron W. Tarajos}

    \textit{\today}
    \vspace{14pt}
\end{center}

\hrule

\section*{Velocity at time $t$}
The change in velocity, $d\nu$, is equal to the product of acceleration, $a$, and the change in time $dt$.
\[
	d\nu = a dt
\]
and we integrate to find the equation to solve for velocity;
\begin{align*}
	d\nu &= adt \\
	\int d\nu &= \int adt \\
	\int d\nu &= a \int dt \\
	\nu &= at + c \\
	\nu_0 &= a(0) + c = c \\
	\nu &= at + \nu_0
\end{align*}
Note that there is no constant $c$ for the integration of $d\nu$ because it is more a symbolic recovering of the velocity function and any constant would be inherently included in the velocity function.

\section*{Position at time $t$}
Similar to accceleration we integrate velocity to find the equation to solve for position.
\begin{align*}
	dx &= \nu dt \\
	\int dx &= \int \nu dt \\
	&= \int \left( \nu_0 + at \right) dt \\
	&= \nu_0 \int dt + a \int tdt \\
	x &= \nu_0t + \frac{1}{2} at^2 + c \\
	x-x_0 &= \nu_0t + \frac{1}{2} at^2 \\
\end{align*}

\section*{An equation without time}
We can solve problems for a various circumstances where one of these variables are missing from the problem entirely using the two previous derivations, starting with an equation without time. Given
\[
	x - x_0 = \nu_0 t + \frac{1}{2} at^2
\]
and
\[
	\nu = \nu_0 + at
\]
We start with the second equation to solve for time as a function of initial velocity, velocity and acceleration;
\[
	\frac{\nu - \nu_0}{a} = t
\]
Furthermore, because acceleration is constant we know that average velocity, $\bar{\nu}$, is
\[
	\bar{\nu} = \frac{\nu + \nu_0}{2}
\]
as well as
\[
	x = \bar{\nu}t + x_0
\]
substituting our equations for time and average velocity;

\begin{align*}
	\left( \frac{\nu + \nu_0}{2} \right) \left( \frac{\nu - \nu_0}{a} \right) + x_0 &= x \\
	\frac{\nu^2 - \nu_o^2}{2a} + x_0 &= x \\
	\nu^2 = 2a \left( x - x_0 \right) + \nu_0^2
\end{align*}

\section*{An equation without acceleration}
Given

\[
	x = x_o + \bar{\nu}t
\]
and

\[
	\bar{\nu} = \frac{\nu + \nu_0}{2}
\]
we obtain

\[
	x - x_0 = \frac{\nu + \nu_0}{2} t
\]

\section*{An equation without initial velocity}
Given
\[
	\nu_0 = \nu - at
\]
We substitute this into the equation for change in position
\[
	x - x_0 = \frac{\left( \nu + \nu - at \right) t}{2}
\]
\[
	x - x_0 = \frac{2\nu t - at^2}{2}
\]
\[
	x - x_0 = \nu t - \frac{at^2}{2}
\]

\end{document}
