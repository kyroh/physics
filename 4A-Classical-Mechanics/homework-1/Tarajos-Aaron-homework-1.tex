\documentclass{article}
\usepackage{graphicx}
\usepackage{amsmath}
\usepackage{pgfplots}
\usepackage{siunitx}
\usepackage{cancel}

\newcommand{\mile}{\text{mi}}
\newcommand{\gallon}{\text{gal}}
\newcommand{\kilo}{\text{k}}
\newcommand{\liter}{\text{L}}
\newcommand{\meter}{\text{m}}
\newcommand{\second}{\text{s}}
\newcommand{\foot}{\text{ft}}

\pgfplotsset{compat=1.18}

\usepackage[a4paper, top=1cm, bottom=2cm, left=2cm, right=2cm, includehead, includefoot]{geometry}

\begin{document}

\noindent
Physics 4A - Classical Mechanics \hfill Prof. Roger King

\noindent\rule{\textwidth}{0.4pt}

\begin{center}
    \textbf{\LARGE Homework 1} \\
    \vspace{12pt}
    \large Aaron W. Tarajos \\
    \textit{\today}
\end{center}

\noindent\rule{\textwidth}{0.4pt}

\section*{Problem 1}
The fuel consumption of cars is specified in Europe in terms of liters per 100 km. Convert
30 miles per gallon to this unit. Note that 1 gallon (U.S.) = 3.79 L.

\subsection*{Solution}
We are given 30 \si{{\mile\per\gallon}} to covert to \si{{\litre\per100\kilo\meter}}. The desired units in some sense are the reciprocal of the given units--volume unit of fuel per distance unit traveled compared to distance unit traveled per unit of fuel.
\[
	30 \frac{\mile}{\gallon} = \frac{1}{30} \frac{\text{gal}}{\mile}
\]
Then we use chain multiplaction to convert miles to kilometers and gallons to liters;
\begin{align*}
	\frac{1}{30} \frac{\cancel{\gallon}}{\mile} \cdot \frac{3.79}{1} \frac{\liter}{\cancel{\gallon}} &= \frac{3.79}{30} \frac{\liter}{\mile} \\
	\frac{3.79}{30} \frac{\liter}{\cancel{\mile}} \cdot \frac{1}{1.609} \frac{\cancel{\mile}}{\kilo\meter} &= \frac{3.79}{48.27} \frac{\liter}{\kilo\meter} \\
	\frac{3.79}{48.27} \frac{\liter}{\kilo\meter} = 0.0785 \frac{\liter}{\kilo\meter} &= 7.85 \frac{\liter}{100 \kilo\meter}
\end{align*}

\section*{Problem 2}
Check the following for dimensional consistency where $t$ is time (s), $\nu$ is speed (m/s), $a$ is acceleration (m/s$^2$), and $x$ is position (m):
\begin{enumerate}
	\item $x = \frac{\nu^2}{2a}$
	\item $x = \frac{1}{2}at$
	\item $t = \sqrt{\frac{2x}{a}}$
\end{enumerate}

\subsection*{Solution}
Checking the dimensional consistency effectively means checking the arithmetic of the units used for each quantity. Meaning that for example the first equation;
\[
	x = \frac{\nu^2}{2a}
\]
the units of $\nu^2$ and $2a$ must cancel such that we are left with just meters, $\meter$, and we find that they do;
\[
	\frac{\left( \meter/\second \right)^2}{\meter/\second^2} = \frac{\meter^{\cancel{2}^1}}{\cancel{\second^2}} \cdot \frac{\cancel{\second^2}}{\cancel{\meter}} = \meter
\]
Therefore, equation 1 is dimensionally consistent. For equation two, we are again looking for position, m, and find that it is not dimensionally consistent;

\[
	\frac{\meter}{\second^2} \cdot \second \ne \meter
\]
For equation three, we are looking for time in seconds, s and find that it is dimensionally consistent;

\[
	\sqrt{\frac{m}{\meter/\second^2}} = \cancel{\meter^{1/2}} \cdot \frac{\second^{2 \cdot 1/2}}{\cancel{\meter^{1/2}}} = s
\]

\section*{Problem 3}
A can of paint that covers 20.0 m$^2$ costs \$24.60. The walls of a room 13.0 ft x 18.0 ft are
8.00 ft high. What is the cost of paint for the walls?

\subsection*{Solution}
If the room is 13 ft x 18 ft then it is

\[
	13\cancel{\foot} \cdot \frac{0.3048}{1} \frac{\meter}{\cancel{\foot}} = 3.9624 \meter
\]
by 5.4864m with 2.4383m high walls and the total surface area to cover is the sum of the surface area of each wall.

\[
	\text{area} = 2\left( 3.9624 \meter \cdot 2.4383 \meter \right) + 2 \left( 5.4864 \meter \cdot 2.4383 \meter \right) = 46.078 \meter^2
\]
and so we will need to cover to buy 3 cans of paint to cover the walls costing us \$73.8. Suppose we are lucky and the store agrees to refund the remaining paint in the partially used can, meters painted per dollar spent;
\[
	\frac{24.60}{20} \frac{\$}{\meter^2} = 1.23 \frac{\$}{\meter^2}
\]
and with 46.078 m$^2$ to paint we have a cost of
\[
	46.078 \meter^2 \cdot 1.23 \frac{\$}{\meter^2} = \$56.68
\]

\section*{Problem 4}
Consider a race car on a $5.00\ \unit{\kilo\meter}$ track. Car A finishes the race in $4.00 \unit{\hour}$ and is 1.50 laps ahead of B at this time. What is B's time for the race?

\subsection*{Solution}
In order to find B's time for the race we need to know the velocity which we can derive from Car A's velocity given that we know how far ahead Car A was when it finished the race as well as how long it took them to finish the race.

\end{document}
