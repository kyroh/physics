\documentclass{article}
\usepackage{graphicx}
\usepackage{amsmath}
\usepackage{pgfplots}

\usepackage[a4paper, top=1cm, bottom=2cm, left=2cm, right=2cm, includehead, includefoot]{geometry}

\pgfplotsset{compat=1.18}

\begin{document}
\flushleft
Physics 4A - Classical Mechanics \hfill Prof. Roger King\\
\hrule

\begin{center}
    \vspace{14pt}
    \textbf{\LARGE Reading notes - Motion Along a Straight Line} \\
    \vspace{12pt}
    \text{\large Aaron W. Tarajos}

    \textit{\today}
    \vspace{14pt}
\end{center}

\hrule

\section*{2.1 Position, Displacement, and Average Velocity}
\subsection*{key ideas}
\begin{enumerate}
	\item The position $x$ of a particle on an $x$ axis locates the particle with respect to the origin
	\item The sign of the position indicates the direction the particle is located with respect to the origin
	\item Displacement $\Delta x$ is the change in position of the particle
		\[
			\Delta x = x_2-x_1
		\]
	\item Average velocity is the displacement of a particle over the time interval
		\[
			\nu_{avg} = \frac{\Delta x}{\Delta t} = \frac{x_2 - x_1}{t_2 - t_1}
		\]
	\item The sign of $\nu_{avg}$ indicates the direction of the motion with respect to the origin. Average velocity is not a function of distance traveled it is a function of initial and final position.
	\item On a graph of $x$ and $t$, average velocity is the slope of the line connecting two points on the graph.
	\item Average speed, $s_{avg}$ is the total distance travel over the time interval
		\[
			s_{avg} = \frac{\text{total distance}}{\Delta t}
		\]
\end{enumerate}

\subsection*{2.1.1 Motion}
Kinematics is the classification and comparison of motions. Initially we restrict motion in a couple of way to examine the most basic cases and get comfortable with the fundamental concepts of kinematics. First, motion is only along a straight line. Forces cause motion but will not be discussed we are looking exclusively at the motion and how it changes. The object in motion is either a particle, point-like object like electrons, or objects that are rigid enough to behave like particles.

\subsection*{2.1.2 Position and Displacement}


\end{document}
