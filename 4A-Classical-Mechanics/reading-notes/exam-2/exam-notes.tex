\documentclass{article}
\usepackage{graphicx}
\usepackage{amsmath}
\usepackage{pgfplots}
\usepackage{physics}
\usepackage{cancel}
\usepackage{enumitem}
\usepackage{txfonts}
\usepackage{multicol}

\pgfplotsset{compat=1.18}

\usepackage[a4paper, top=1cm, bottom=1cm, left=1cm, right=1cm, includehead, includefoot]{geometry}

\begin{document}

\noindent
Physics 4A - Classical Mechanics \hfill Aaron W. Tarajos
\begin{center}
	\textbf{Exam 2 Notes}
\end{center}

\noindent\rule{\textwidth}{0.4pt}

\begin{multicols}{2}
\flushleft
polar $\to$ component: $a_x = a\cos\theta \quad \text{and} \quad a_y = a\sin\theta$\\
component $\to$ polar: $a = \sqrt{a_x^2 + a_y^2} \quad \text{and} \quad \tan\theta = \frac{a_y}{a_x}$
\subsection*{Friction}
Static friction is equal to the opposing force, up to a maximum given by;
\[
	F_{s,\text{max}} = \mu_s F_n
\]
after the force exceeds the maximum force of static friction the object slips where kinetic friction acts on the object such that
\[
	F_k = \mu_k F_n
\]

\subsection*{Uniform circular motion}
\begin{align*}
	a &= \frac{v^2}{r} \\
	F_c &= \frac{mv^2}{r}
\end{align*}

\subsection*{Work and energy}
Kinetic energy;
\[
	K = \frac{1}{2}mv^2
\]
Work;
\[
	W = Fd \cos \phi \quad \text{Or} \quad W = \Delta K
\]
work is defined as the dot product of Force and Distance. Work done by gravity;
\[
	W_g = mgd \cos \phi
\]
Work done by a spring;
\[
	W_s = -\frac{1}{2}kx^2
\]
Hooke's law;
\[
	F_s = -kx
\]
Potential energy;
\[
	\Delta U = -W
\]
gravitational potential energy;
\[
	U = mgy
\]
spring potential energy;
\[
	U = \frac{1}{2}kx^2
\]
For non-linear forces;
\[
	W = \int_{x_1}^{x_2} F(x)\ dx
\]

\subsection*{Power}
\[
	P = \frac{dW}{dt} \quad \text{or} \quad P = \vec{F} \cdot \vec{v} = Fv \cos \phi
\]

\subsection*{Mechanical Energy}
\[
	E_\text{mech} = K + U
\]
Unless energy enters the system energy is conserved;
\[
	\Delta E_\text{mech} = \Delta K + \Delta U = 0
\]
kinetic energy can become potential energy or the other way around. Kinetic energy is never negative.\\
Therma energy is
\[
	E_\text{th} = F_k \cdot d
\]

\subsection*{Work from external forces}
Work can be done by external forces influencing the system;
\[
	W = \Delta K + \Delta U \quad \text{Or} \quad W = \Delta E
\]

\subsection*{Center of Mass}
For many particle system;
\[
	x_{CM} = \frac{1}{M}\sum_{i=1}^n m_i x_i
\]
where $M$ is the total mass of all the particles. (Average location $x$ weighted by mass $m$, i.e. expected value).
For solid bodies;
\[
	x_cm = \frac{1}{V} \int x w(x) \ dx
\]
where $w$ is width as a function of $x$ and $V$ is volume. Just add an integral for each dimension.

\subsection*{Momentum}
Linear momentum;
\[
	p = m \vec{v}
\]
for a system of object total momentum is just the sum of $p_i$.
Impulse;
\[
	\vec{J} = \Delta \vec{p}
\]
\[
	\frac{d\vec{p}}{dt} = \vec{F}
\]
Momentum is conserved (one of the fundamental conservation laws.

\end{multicols}

\begin{multicols}{2}
\subsection*{Collisions}
Perfectly elastic $\implies$ kinetic energy is conserved \\
Inelastic $\implies$ kinetic energy is not conserved \\
Perfectly inelastic $\implies$ kinetic energy is not conserved\\
Conservation of momentum;
\[
	m_1v_{1,i} + m_2v_{2,i} = m_1v_{1,f} + m_2v_{2,f}
\]
One-dimensional inelastic collisions;
\[
	V= \frac{m_1}{m_1 + m_2}v_{1,i}
\]
where $V$ is the final velocity of the total system.\\
One-dimensional elastic collisions;
\[
	\frac{1}{2}m_1v_{1,i} = \frac{1}{2}m_1v_{1,f} + \frac{1}{2}m_2v_{2,f}
\]
Then we can solve for various quantities using;
\[
	v_{1,f} = \frac{m_1 - m_2}{m_1 + m_2}v_{1,i}
\]
\[
	v_{2,f} = \frac{2m_1}{m_1 + m_2}v_{1,i}
\]

\subsection*{Angular kinematics}
It works the same as linear motion but with different variables. Position is defined by $\theta$ in radians;
\[
	\theta = \frac{S}{r}
\]
angular velocity is
\[
	\omega = \frac{d\theta}{dt}
\]
angular acceleration is
\[
	\alpha = \frac{d^2\theta^2}{dt^2}
\]
kinematic equations \\
    \begin{tabular}{|l|l|}
        \hline
        Variable & Equation \\
        \hline
        Velocity & $v = at + v_0$ \\
        Position & $\Delta x = v_0t + \frac{1}{2}at^2$ \\
        Missing $t$ & $v^2 = 2a\Delta x + v_0^2$ \\
        Missing $a$ & $\Delta x = \frac{v +v_0}{2}t$ \\
        Missing $v_0$ & $\Delta x = vt - \frac{1}{2}at^2$ \\
        \hline
    \end{tabular}\\
other useful stuff;
\begin{align*}
	v_t &= \omega r \\
	a_t &= \alpha r
\end{align*}
for tangential acceleration and velocity. \\
Centripetal acceleration;
\[
	a_c = \omega^2 r
\]
Period of revolution;
\[
	T = \frac{2 \pi r}{v} = \frac{2 \pi}{\omega}
\]

\begin{tabular}{|l|l|}
	\hline
	Variable & Equation \\
	\hline
	Horizontal displacement & $\Delta x = v_0 \cos \theta t$ \\
	Vertical displacement & $\Delta y = v_0 \sin \theta t - \frac{1}{2}gt^2$ \\
	Vertical velocity & $v_y = v_0 \sin \theta - gt$ \\
	Trajectory & $\Delta y = \tan \theta \Delta x - \frac{g \Delta x^2}{2(v_0\cos \theta)^2}$ \\
	Range & $R = \frac{v_0^2 \sin 2\theta}{g}$ \\
	Rolling ball incline & $a = -\frac{g \sin \theta}{1 + I_\text{com} / MR^2}$ \\
	\hline
\end{tabular}

\subsection*{Oscillations}
\[
	x = x_m \cos(\omega t + \phi)
\]
\[
	\omega = \sqrt{\frac{k}{m}}
\]
\[
	T = 2 \pi \sqrt{\frac{m}{k}} \quad \text{and} \quad T = \frac{1}{f}
\]

\subsection*{Harmonic motion}
\[
	\tau = - \kappa \theta \implies I \alpha = - \kappa \theta
\]
for a simple pendulum
\[
	\omega = \sqrt{\frac{g}{L}}
\]
physical pendulum
\[
	\omega = \sqrt{\frac{mgL}{I}}
\]
and
\[
	I = I_\text{com} + mR^2
\]
dampening force
\[
	F_d = -bv \quad \text{and} \quad x(t) = e^\frac{-bt}{2m} \cos (\omega^\prime)t + \phi
\]
where
\[
	\omega^\prime = \sqrt{\frac{k}{m}-\frac{b^2}{4m^2}}
\]

\subsection*{Useful units}
Work: J $\rightarrow$ kg m$^2$/s$^2$ \\
Impulse: N$\cdot$s $\rightarrow$ kg m/s \\
Force: N $\rightarrow$ kg m/s$^2$


\subsection*{Check your algebra you fucking idiot}
\subsection*{Potential tricks}
\begin{itemize}
	\item Gravitational work require change in height i.e. satelite orbiting earth has no gravitational work.
	\item The location of a pendulum matters for tension because of gravity. $\frac{mv^2}{r} = F_t + mg\cos \phi$
	\item Double check angles for work, may need to add or subtract 180$\circ$ depending on orientation
	\item When in doubt use calculus and see if the result makes physical sense.
	\item if Velocity is constant
\end{itemize}

\end{multicols}


\end{document}
