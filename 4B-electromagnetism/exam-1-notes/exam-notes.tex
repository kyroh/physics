\documentclass{article}
\usepackage{graphicx}
\usepackage{amsmath}
\usepackage{pgfplots}
\usepackage{physics}
\usepackage{cancel}
\usepackage{enumitem}
\usepackage{txfonts}
\usepackage{multicol}

\pgfplotsset{compat=1.18}

\usepackage[a4paper, top=1cm, bottom=1cm, left=1cm, right=1cm, includehead, includefoot]{geometry}

\begin{document}

\noindent
Physics 4B - Electromagnetism \hfill Aaron W. Tarajos
\begin{center}
	\textbf{Exam 1 Notes}
\end{center}

\noindent\rule{\textwidth}{0.4pt}

\begin{multicols}{2}
\flushleft
polar $\to$ component: $a_x = a\cos\theta \quad \text{and} \quad a_y = a\sin\theta$\\
component $\to$ polar: $a = \sqrt{a_x^2 + a_y^2} \quad \text{and} \quad \tan\theta = \frac{a_y}{a_x}$

\subsection*{Angular kinematics}
It works the same as linear motion but with different variables. Position is defined by $\theta$ in radians;
\[
	\theta = \frac{S}{r}
\]
angular velocity is
\[
	\omega = \frac{d\theta}{dt}
\]
angular acceleration is
\[
	\alpha = \frac{d^2\theta^2}{dt^2}
\]
kinematic equations \\
    \begin{tabular}{|l|l|}
        \hline
        Variable & Equation \\
        \hline
        Velocity & $v = at + v_0$ \\
        Position & $\Delta x = v_0t + \frac{1}{2}at^2$ \\
        Missing $t$ & $v^2 = 2a\Delta x + v_0^2$ \\
        Missing $a$ & $\Delta x = \frac{v +v_0}{2}t$ \\
        Missing $v_0$ & $\Delta x = vt - \frac{1}{2}at^2$ \\
        \hline
    \end{tabular}\\
other useful stuff;
\begin{align*}
	v_t &= \omega r \\
	a_t &= \alpha r
\end{align*}
for tangential acceleration and velocity. \\
Centripetal acceleration;
\[
	a_c = \omega^2 r
\]
Period of revolution;
\[
	T = \frac{2 \pi r}{v} = \frac{2 \pi}{\omega}
\]

\begin{tabular}{|l|l|}
	\hline
	Variable & Equation \\
	\hline
	Horizontal displacement & $\Delta x = v_0 \cos \theta t$ \\
	Vertical displacement & $\Delta y = v_0 \sin \theta t - \frac{1}{2}gt^2$ \\
	Vertical velocity & $v_y = v_0 \sin \theta - gt$ \\
	Trajectory & $\Delta y = \tan \theta \Delta x - \frac{g \Delta x^2}{2(v_0\cos \theta)^2}$ \\
	Range & $R = \frac{v_0^2 \sin 2\theta}{g}$ \\
	\hline
\end{tabular}


\subsection*{Useful units}
Work: J $\rightarrow$ kg m$^2$/s$^2$ \\
Impulse: N$\cdot$s $\rightarrow$ kg m/s \\
Force: N $\rightarrow$ kg m/s$^2$


\subsection*{Check your algebra you fucking idiot}

\subsection*{Electrostatic Force}
Coulomb's Law
\[
	\vec{F} = k\frac{q_1q_2}{r^2}
\]

\subsection*{Fields}
Electric field for a point is just force but divide out $q_1$.\\
Electric field due to a line of charge (point perpendicular to the end);
\begin{align*}
	\vec{E} &= \int_{-L}^0 k \frac{dq}{r^2}\ \vu r\\
			&= \int_{-L}^0 k \frac{dq}{x^2 + z^2}\ \vu r \\
			&= \int_{-L}^0 k \frac{\lambda dx}{x^2+z^2}\ \vu r \\
			&= \int_{-L}^0 k \frac{\lambda dx}{x^2+z^2}\ \frac{-x\ \vu i + z\ \vu j}{\sqrt{x^2+z^2}} \\
			&= k\lambda \left[ \frac{1}{\sqrt{z^2}}-\frac{1}{\sqrt{L^2+z^2}} \right]\ \vu i + \frac{k \lambda L}{z\left(L^2+z^2\right)^{1/2}}\ \vu j
\end{align*}
Electric field due to a disc;
\begin{align*}
	\vec{E} &= \int d\vec{E} \\
			&= \int \frac{1}{4\pi\epsilon_0} \frac{dq}{r^2 + z^2}\ \vu r \\
			&= \frac{1}{4\pi\epsilon_0} \int \frac{\sigma \cdot dA}{r^2+z^2}\ \vu r \\
			&= \frac{1}{4\pi\epsilon_0} \int \frac{\sigma \cdot (2\pi r dr)}{r^2+z^2}\ \vu r \\
			&= \frac{1}{4\pi\epsilon_0} \int \frac{\sigma \cdot (2r dr)}{\left(r^2+z^2\right)^{3/2}}\ z \vu j \\
			&= \frac{\sigma z}{4\pi\epsilon_0} \int_0^R \frac{2r dr}{\left(r^2+z^2\right)^{3/2}} \vu j \\
			&= \frac{\sigma}{2 \epsilon_0} \left(1 - \frac{z}{\sqrt{z^2+R^2}} \right)
\end{align*}
\textbf{Charge Density:} $dq = \lambda ds$ for a line, $dq = \sigma dA$ for a sufrace, etc... and these are simply $\frac{\text{charge}}{\text{quantity}}$ \\
Electric Field due to a dipole;
\[
	\vec{E} = \frac{2kQd}{z^3\left(1-d^2/4z^2\right)^2}\ \vu k
\]

\subsection*{Dipole stuff}
\begin{align*}
	\vec{p} &= Qd\vu k \\
	\vec{\tau} &= \vec{p} \times \vec{E} \\
	U &= -\vec{p} \cdot \vec{E} \\
	W &= - \Delta U
\end{align*}

\end{multicols}


\end{document}
