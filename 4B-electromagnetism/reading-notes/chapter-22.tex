\documentclass{article}
\usepackage{graphicx}
\usepackage{amsmath}
\usepackage{pgfplots}
\usepackage{tikz}
\usepackage{txfonts}
\usepackage{physics}
\usepackage{hyperref}
\usepackage[a4paper, top=1cm, bottom=2cm, left=2cm, right=2cm, includehead, includefoot]{geometry}
\pgfplotsset{compat=1.18}

\begin{document}
\noindent
Physics 4B - Electromagnetism \hfill Prof. Alfred Cauthen

\noindent\rule{\textwidth}{0.4pt}

\begin{center}
    \textbf{\LARGE Chapter 21 - Coulomb's Law} \\
    \vspace{12pt}
    \large Aaron W. Tarajos \\
    \textit{\today}
\end{center}

\noindent\rule{\textwidth}{0.4pt}

\section*{22.1 Electric Field}
\subsection*{Key Ideas}
\begin{itemize}
    \item An electric field is a vector field that describes the force per unit charge experienced by a small positive test charge.
    \item The electric field \( \mathbf{E} \) is defined as:
    \[
    \mathbf{E} = \frac{\mathbf{F}}{q_0}
    \]
    where \( \mathbf{F} \) is the force experienced by the test charge \( q_0 \).
    \item Field lines provide a visualization of electric fields:
    \begin{itemize}
        \item Lines originate on positive charges and terminate on negative charges.
        \item The density of lines corresponds to the field's magnitude.
    \end{itemize}
\end{itemize}

\subsection*{Field of a Point Charge}
The electric field due to a point charge \( q \) at distance \( r \) is:
\[
E = \frac{1}{4 \pi \epsilon_0} \frac{|q|}{r^2}
\]
where:
\begin{itemize}
    \item \( \epsilon_0 = 8.85 \times 10^{-12} \, \text{C}^2 / \text{N·m}^2 \) is the permittivity of free space.
\end{itemize}

\section*{22.2 Electric Dipoles}
\subsection*{Definition}
\begin{itemize}
    \item An electric dipole consists of two charges of equal magnitude but opposite signs, separated by a distance \( d \).
    \item The dipole moment \( \mathbf{p} \) is defined as:
    \[
    \mathbf{p} = q \cdot d
    \]
\end{itemize}

\subsection*{Electric Field of a Dipole}
The field of a dipole at a distant point along its axis is:
\[
E = \frac{1}{2 \pi \epsilon_0} \frac{p}{z^3}
\]
where \( z \) is the distance from the dipole's center.

\section*{22.3 Continuous Charge Distributions}
\subsection*{Key Concept}
The electric field due to a continuous charge distribution is found by integrating the fields of infinitesimal charge elements:
\[
\mathbf{E} = \int \frac{1}{4 \pi \epsilon_0} \frac{\mathrm{d}q}{r^2} \hat{r}
\]

\section*{22.4 Applications and Examples}
\subsection*{Force on a Charge in an Electric Field}
The force on a charge \( q \) in a field \( \mathbf{E} \) is:
\[
\mathbf{F} = q \mathbf{E}
\]

\subsection*{Torque on a Dipole}
An electric dipole in a field \( \mathbf{E} \) experiences a torque:
\[
\boldsymbol{\tau} = \mathbf{p} \times \mathbf{E}
\]
The potential energy associated with the dipole is:
\[
U = -\mathbf{p} \cdot \mathbf{E}
\]

\section*{Summary}
\begin{itemize}
    \item Electric fields describe how charges interact at a distance.
    \item Coulomb’s Law and superposition principles are fundamental in calculating electric fields.
    \item Fields from continuous distributions require integration.
\end{itemize}

\end{document}
