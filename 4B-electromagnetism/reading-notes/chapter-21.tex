\documentclass{article}
\usepackage{graphicx}
\usepackage{amsmath}
\usepackage{pgfplots}
\usepackage{tikz}
\usepackage{txfonts}
\usepackage{physics}
\usepackage{hyperref}
\usepackage[a4paper, top=1cm, bottom=2cm, left=2cm, right=2cm, includehead, includefoot]{geometry}
\pgfplotsset{compat=1.18}

\begin{document}
\noindent
Physics 4B - Electromagnetism \hfill Prof. Alfred Cauthen

\noindent\rule{\textwidth}{0.4pt}

\begin{center}
    \textbf{\LARGE Chapter 21 - Coulomb's Law} \\
    \vspace{12pt}
    \large Aaron W. Tarajos \\
    \textit{\today}
\end{center}

\noindent\rule{\textwidth}{0.4pt}

\section*{21.1 Electric Charge}
\subsection*{Key Ideas}
\begin{itemize}
    \item Electric charge is a fundamental property of matter that causes it to experience a force in an electric field.
    \item Charges are either positive or negative, with like charges repelling and opposite charges attracting.
    \item The unit of charge is the Coulomb (C), and the elementary charge is \( e = 1.602 \times 10^{-19} \, \text{C} \).
\end{itemize}

\subsection*{Charge Quantization and Conservation}
\textbf{Quantization:} Electric charge exists in discrete packets of \( n \cdot e \), where \( n \) is an integer.\\
\textbf{Conservation:} The total charge in an isolated system is constant, even as charges are transferred between objects.

\section*{21.2 Coulomb's Law}
\subsection*{The Force Between Two Charges}
Coulomb's Law describes the magnitude of the force between two point charges:
\[
\mathbf{F} = k \frac{q_1 q_2}{r^2} \hat{r}
\]
where:
\begin{itemize}
    \item \( k = \frac{1}{4\pi \epsilon_0} = 8.99 \times 10^9 \, \text{N·m}^2/\text{C}^2 \),
    \item \( q_1, q_2 \) are the charges,
    \item \( r \) is the distance between the charges,
    \item \( \hat{r} \) is the unit vector along the line joining the charges.
\end{itemize}

\subsection*{Nature of the Force}
\begin{itemize}
    \item The force is attractive if \( q_1 q_2 < 0 \) and repulsive if \( q_1 q_2 > 0 \).
    \item The force acts along the line connecting the two charges.
\end{itemize}

\section*{21.3 Superposition Principle}
If multiple charges exert forces on a test charge, the net force is the vector sum of all individual forces:
\[
\mathbf{F}_{\text{net}} = \sum \mathbf{F}_i
\]

\subsection*{Example Problem}
Three charges are arranged in a triangle. Calculate the net force on one charge using the superposition principle:
\[
\mathbf{F}_{\text{net}} = \mathbf{F}_1 + \mathbf{F}_2
\]

\section*{21.4 Analogies with Gravitational Force}
\begin{itemize}
    \item Coulomb's Law is analogous to Newton's Law of Gravitation:
    \[
    \mathbf{F}_g = G \frac{m_1 m_2}{r^2}
    \]
    where \( G \) is the gravitational constant.
    \item Differences:
    \begin{itemize}
        \item Gravitational forces are always attractive, while electrostatic forces can be attractive or repulsive.
        \item Gravitational forces act between masses; Coulomb forces act between charges.
    \end{itemize}
\end{itemize}

\section*{21.5 Applications of Coulomb's Law}
\subsection*{Shell Theorem}
\begin{itemize}
    \item A charged particle outside a spherical shell of charge behaves as if all the charge were concentrated at the shell's center.
    \item Inside a spherical shell, the net electrostatic force on a charged particle is zero.
\end{itemize}

\section*{Summary}
\begin{itemize}
    \item Coulomb's Law quantifies the electrostatic force between charges.
    \item The principle of superposition allows calculation of net forces in systems with multiple charges.
    \item Conservation and quantization of charge are fundamental principles of electromagnetism.
\end{itemize}

\end{document}
