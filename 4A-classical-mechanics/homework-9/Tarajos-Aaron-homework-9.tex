\documentclass{article}
\usepackage{graphicx}
\usepackage{amsmath}
\usepackage{pgfplots}
\usepackage{physics}
\usepackage{cancel}
\usepackage{enumitem}
\usepackage{txfonts}

\pgfplotsset{compat=1.18}

\usepackage[a4paper, top=1cm, bottom=2cm, left=2cm, right=2cm, includehead, includefoot]{geometry}

\begin{document}

\noindent
Physics 4A - Classical Mechanics \hfill Prof. Roger King

\noindent\rule{\textwidth}{0.4pt}

\begin{center}
    \textbf{\LARGE Homework 9} \\
    \vspace{12pt}
    \large Aaron W. Tarajos \\
    \textit{\today}
\end{center}

\noindent\rule{\textwidth}{0.4pt}

\section*{Problem 1}
A spring gun with k = 90.0 N/m is compressed by 5 cm. What is the exit speed of a 2.10-g projectile?

\subsection*{Solution}
Let $U$ be potential energy, then;
\[
	U = \frac{1}{2} kx^2
\]
and $K$ be kinetic energy, then we have;
\begin{align*}
	U &= K \\
	\frac{1}{2}kx^2 &= \frac{1}{2}mv^2 \\
	v^2 &= \frac{kx^2}{m} \\
	v &= \sqrt{\frac{kx^2}{m}}
\end{align*}
For the given values;
\[
	v = \sqrt{\frac{(90.0)(0.05^2)}{0.00210}} = \boxed{10.351\ \text{m}/\text{s}}
\]

\section*{Problem 2}
The United States, with a population of $2.2 \times 10^8$ people, consumes $5 \times 10^{19}$ J per year. \\
(a)What is the per capita consumption in watts? \\
(b) The sun's radiation provides the earth with 1000 W/m$^2$.
Assuming solar energy can be converted to electrical energy with a 20\% efficiency,
how much area is needed to serve the energy needs of each U.S. citizen?

\subsection*{Solution}
\subsubsection*{Part a:}
Let $E_T$ be the total power consumed in the U.S. in Joules
\[
	E_T = 5 \times 10^{19} \cdot \frac{1}{365} \cdot \frac{1}{24} \cdot \frac{1}{3600} = 1.585 \times 10^{12}\ \text{W}
\]
then the per capita energy consumption is $E_C$;
\[
	E_C = \frac{E_T}{2.2 \times 10^8} = \boxed{7206.771\ \text{W}}
\]

\subsubsection*{Part b:}
Let $A$ be the area needed to serve the energy needs of each citizen;
\[
	A = \frac{E_C}{1000 \cdot 0.2} = \boxed{36.034\ \text{m}^2}
\]


\section*{Problem 3}
A 0.595-kg object is released from a height of 3.60 m and lands on the ground. Find: \\
(a) the work done by gravity; \\
(b) the change in kinetic energy of the ball; \\
(c) the speed just before it lands using energy methods. Ignore air resistance.

\subsection*{Solution}
\subsubsection*{Part a:}
\[
	W = mgd = (0.595)(9.81)(3.60) = \boxed{21.013\ \text{J}}
\]

\subsubsection*{Part b:}
\[
	\Delta K = W = \boxed{21.013\ \text{J}}
\]

\subsubsection*{Part c:}
\begin{align*}
	W &= \Delta K \\
	W &= \frac{1}{2}mv^2 - \frac{1}{2}mv_0^2 \\
	W &= \frac{1}{2}mv^2 - \frac{1}{2}m(0)^2 \\
	mgd &= \frac{1}{2}mv^2 \\
	v^2 &= 2gd \\
	v &= \sqrt{2gd} = \boxed{8.404\ \text{m}/\text{s}}
\end{align*}



\section*{Problem 4}
Two horses pull a barge along a canal at a steady 5.00 km/h, as shown in the figure. The tension in each rope
is 420 N and each is at 30$^\circ$ to the direction of motion. What is the horsepower provided by the horses?

\subsection*{Solution}
The horses are walking at constant velocity which means that the net force in the $x$ direction is zero. Therefore
\begin{align*}
	F_T \cos \theta - F_x &= 0 \\
	F_T \cos \theta &= F_x \\
	F_x &= 420 \cos 30 = 363.730\ \text{N}
\end{align*}
There are two horses so the force is 2 times that, then power is given by
\[
	P = F_x v = \frac{2 \left(420 \cos 30 \right) \cdot 1.3889}{745.7} = \boxed{1.355\ \text{HP}}
\]

\section*{Problem 5}
A pendulum bob of mass 0.710 kg is suspended by a string of length 1.50 m. The bob is released from rest
when the string is at 30$^\circ$ to the vertical. The swing is interrupted by a peg 1.00 m vertically below the support
as shown below. What is the maximum angle to the vertical made by the string after it hits the peg?

\subsection*{Solution}
The potential energy before the pendulum is dropped is given by
\[
	U_1 = mgh_1
\]
where $h_1$ is the vertical distance that the pendulum falls before the string hits the peg;
\[
	h_1 = r_1 - r_1\cos\theta
\]
giving the equation;

\[
	U_1 = mg\left(r_1 - r_1\cos\theta\right)
\]
similarly, the potential energy when the pendulum is at its maximum height on the other side of the peg is given by;
\[
	U_2 = mg \left(r_2 - r_2 \cos\phi \right)
\]
and the potential energy in both cases must be equal, therefore;
\begin{align*}
	U_1 &= U_2 \\
	mg\left(r_1 - r_1\cos\theta\right) &= mg \left(r_2 - r_2 \cos\phi \right) \\
	r_2 \cos \phi &= r_1\cos\theta - r_1 + r_2 \\
	\cos\phi &= \frac{r_1\cos\theta - r_1 + r_2}{r_2} \\
	\phi &= \arccos \left( \frac{r_1\cos\theta - r_1 + r_2}{r_2} \right) \\
	\phi &= \arccos \left( \frac{1.5\cos(30) - 1.5 + 0.5}{0.5} \right) = \boxed{53.268^\circ}
\end{align*}


\section*{Problem 6}
A 2.00-kg block slides on a frictionless horizontal surface and is connected on one side to a spring with a
spring constant of 45.0 N/m) as shown the figure. The other side is connected to a 4.00-kg block that hangs
vertically. The system starts from rest with the spring unextended. \\
(a) What is the maximum extension of the spring? \\
(b) What is the speed of the 4.00-kg block when the extension is 50 cm?

\subsection*{Solution}
\subsubsection*{Part a:}
The maximum extension of the spring occurs when the downward force of gravity on the blocks is equal to the force of the spring acting on the blocks;
\begin{align*}
	\frac{1}{2}kd^2 &= m_2gd \\
\end{align*}

\subsubsection*{Part b:}
The speed of the block when the spring is extended 50 cm is;
\begin{align*}
	F &= kd \\
	(m_1 + m_2)a &= kd \\
	a &= \frac{kd}{m_1 + m_2} \\
	\int a &= \int \frac{kd}{m_1 + m_2} \\
	v &= \frac{kd^2}{2\left( m_1 + m_2 \right)} \\
	v &= \frac{45.0 \cdot 0.5^2}{2 \cdot 6.00} = \boxed{0.938\ \text{m}/\text{s}}
\end{align*}

\section*{Problem 7}
A cart with a mass of 3.20 kg, an initial speed of 5.15 m/s and an initial height of 4.00 m is moving
towards a hill of height 5.00 m, as shown in the figure. On the other side of the hill is a spring with a spring
constant of 125 N/m and a height of 2.00 m. \\
(a) Does the trolley reach the spring? \\
(b) If so, what is the maximum compression? Ignore frictional losses and the rotational energy of the wheels.

\section*{Problem 8}
A projectile is fired at 27.0 m/s in a direction 65$^\circ$ above the horizonal from a rooftop of height 40.0 m. Use
energy considerations to find: \\
(a) the speed with which it lands on the ground; \\
(b) the height at which its speed is 15.0 m/s.

\end{document}
