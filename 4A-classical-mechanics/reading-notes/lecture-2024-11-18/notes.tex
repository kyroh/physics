\documentclass{article}
\usepackage{amsmath}
\usepackage{physics}
\usepackage{enumitem}
\usepackage{txfonts}

\usepackage{hyperref}
\hypersetup{
    colorlinks=true,
    linkcolor=blue,
    filecolor=magenta,
    urlcolor=cyan,
    pdftitle={Overleaf Example},
    pdfpagemode=FullScreen,
    }

\urlstyle{same}

\usepackage[a4paper, top=1cm, bottom=2cm, left=2cm, right=2cm, includehead, includefoot]{geometry}

\title{Lecture Notes}
\author{Aaron W. Tarajos}

\begin{document}

\maketitle

\section{Angular Kinematics Review}
Angular position
\[
	\theta(t) = \frac{S}{r}
\]
Angular velocity
\[
	\omega(t) = \frac{d \theta(t)}{dt}
\]
units are just s$^{-1}$ and angular acceleration is;
\[
	\alpha(t) = \frac{d \omega(t)}{dt} = \frac{d^2 \theta(t)}{dt^2}
\]
then tangential velocity is
\[
	\boxed{v_t = r \omega}
\]
and centripetal/radial acceleration is
\[
	a_r = r \alpha \quad \text{or} \quad a_r = \frac{v^2}{r} = r \omega^2
\]

\section{Rotational Interia}
Imagine a rod rotating counter clockwise; $\omega$ then would be positive, in reality $\vec{\omega}$, and the direction would be out of the page. Because we want to model with constant velocity we need a unique direction, therefore we define the direction of angular velocity as a cross product. Some notation
\begin{itemize}
	\item Out of the page $\rightarrow \odot$
	\item Into the page $\rightarrow \otimes$
\end{itemize}

\subsection*{Rotational Kinetic Energy}
In linear motion
\[
	K = \frac{1}{2}mv^2
\]
For some infinitesimal section of the rod, the velocity would be
\[
	v_i = r_i \omega
\]
so then the infinitesimal kinetic energy is;
\[
	K_i  = \frac{1}{2}m_i v_i^2 = \frac{1}{2}m_i r_i^2 \omega_i^2
\]
or
\[
	\sum_i K_i = \frac{1}{2} \omega^2 \sum_i m_i r_i^2
\]
so we define rotational intertia;
\begin{equation}
	I \coloneqq \sum_i m_i r_i^2
\end{equation}
Therefore rotational kinetric energy is given by
\[
	K = \frac{1}{2}I \omega^2
\]
rotational inertia is how hard it is to get something rotating and if it is rotating how hard it is to get it to stop. Time to integrate;
\begin{align*}
	I &= \sum_i m_i r)i^2 \\
	  &= \int r(m)^2\ dm
\end{align*}
Some variables
\begin{align*}
	\rho &= \frac{m}{V} \\
	\sigma &= \frac{m}{A} \\
	\lambda &= \frac{m}{L}
\end{align*}
so we have
\[
	\frac{M}{L} = \frac{dm}{dx} \implies dm = \frac{M}{L}dx
\]
then we use that to construct

\[
	I = \frac{M}{L} \int_{-L/2}^{L/2} x^2 dx
\]
basically use density to substitute $dm$ in the integral and then integrate.

\end{document}
