\documentclass{article}
\usepackage{graphicx}
\usepackage{amsmath}
\usepackage{pgfplots}
\usepackage{physics}
\usepackage{cancel}
\usepackage{enumitem}
\usepackage{txfonts}

\pgfplotsset{compat=1.18}

\usepackage[a4paper, top=1cm, bottom=1cm, left=1cm, right=1cm, includehead, includefoot]{geometry}

\begin{document}

\noindent
Physics 4A - Classical Mechanics \hfill Aaron W. Tarajos
\begin{center}
	\textbf{Exam 1 Cheatsheet}
\end{center}

\noindent\rule{\textwidth}{0.4pt}

\subsubsection*{Kinematics}
\begin{minipage}{0.4\textwidth}
    \begin{tabular}{|l|l|}
        \hline
        Variable & Equation \\
        \hline
        Velocity & $v = at + v_0$ \\
        Position & $\Delta x = v_0t + \frac{1}{2}at^2$ \\
        Missing $t$ & $v^2 = 2a\Delta x + v_0^2$ \\
        Missing $a$ & $\Delta x = \frac{v +v_0}{2}t$ \\
        Missing $v_0$ & $\Delta x = vt - \frac{1}{2}at^2$ \\
        \hline
    \end{tabular}
\end{minipage}
\begin{minipage}{0.4\textwidth}
General guidelines for solving problems
    \begin{enumerate}[left=0pt]
        \item List the known and unknown quantities.
        \item Determine the variable we are solving for and what variables are needed to solve it.
        \item Derive an equation using those given to solve for your variables.
        \item Plugin a numbers and don't forget units.
	\end{enumerate}
\end{minipage}
\hrule
\subsubsection*{Vector algebra}
\begin{itemize}
	\item polar $\to$ component: $a_x = a\cos\theta \quad \text{and} \quad a_y = a\sin\theta$
	\item component $\to$ polar: $a = \sqrt{a_x^2 + a_y^2} \quad \text{and} \quad \tan\theta = \frac{a_y}{a_x}$
	\subitem - note that finding $\arctan\theta$ may need to check orientation of the resulting vector angle
	\item dot product gives a scalar that is the magnitude of a vector $\vb{B}$ in the direction of a vector $\vb{A}$:
	\[
		\vb{A} \cdot \vb{B} = AB \cos\theta \quad \text{OR} \quad \vb{A} \cdot \vb{B} = A_xB_x + A_yB_y + \dots
	\]
	\item scalar product gives a vector that is perpendicular to the two vectors:
	\[
		\vb{A} \times \vb{B} = AB \sin \theta \hat n
	\]
	where $\hat n$ is a unit vector in a direction perpendicular to $\vb{A}$ and $\vb{B}$. For component vectors
	\begin{align*}
		\vb{A} \times \vb{B} &= \det \begin{bmatrix}
			\hat i & \hat j & \hat k \\
			a_x & a_y & a_z \\
			b_x & b_y & b_z
			\end{bmatrix} = \begin{bmatrix} a_y & a_z \\ b_y & b_z\end{bmatrix}\ \hat i + \begin{bmatrix} a_x & a_z \\ b_x & b_z\end{bmatrix}\ \hat j + \begin{bmatrix} a_x & a_y \\ b_x & b_y\end{bmatrix}\ \hat k \\
		&= (a_yb_z - b_ya_z)\ \hat i + (a_xb_z - b_xa_z)\ \hat j + (a_xb_y - b_xa_y)\ \hat k
	\end{align*}
\end{itemize}
\hrule
\subsubsection*{Motion in two dimensions}
\begin{minipage}{0.45\textwidth}
\begin{tabular}{|l|l|}
	\hline
	Variable & Equation \\
	\hline
	Horizontal displacement & $\Delta x = v_0 \cos \theta t$ \\
	Vertical displacement & $\Delta y = v_0 \sin \theta t - \frac{1}{2}gt^2$ \\
	Vertical velocity & $v_y = v_0 \sin \theta - gt$ \\
	Trajectory & $\Delta y = \tan \theta \Delta x - \frac{g \Delta x^2}{2(v_0\cos \theta)^2}$ \\
	Range & $R = \frac{v_0^2 \sin 2\theta}{g}$ \\
	\hline
\end{tabular}
\end{minipage}
\begin{minipage}{0.45\textwidth}
Guidelines for solving problems
	\begin{enumerate}
		\item Simplify the problem to each dimensional component.
		\item Solve each component with the same process as kinematics
		\item Horizontal motion is linear and vertical motion is parabolic because acceleration is zero and a constant respectively.
		\item Equation for range can only be used when $\Delta y = 0$, otherwise use Trajectory and solve for $\Delta x$.
	\end{enumerate}
\end{minipage}
\hrule

\pagebreak
\subsubsection*{Newton's Laws}
\begin{minipage}{0.4\textwidth}
\begin{enumerate}
	\item If $\vec{F}_\text{net} = 0$ then $\vec{a}=0$
	\item $\vec{F}_\text{net} = m \vec{a}$
	\item $\vec{F}_{AB} = - \vec{F}_{BA}$
\end{enumerate}
\end{minipage}
\begin{minipage}{0.5\textwidth}
Annecdotes on applying Newton's Laws
\begin{enumerate}
	\item $\vec{F}_\text{net}$ is comprised of components $F_{\text{net,}x} = m a_x$, $F_{\text{net,}y} = m a_y$, $\vec{a}$ represents the acceleration of the entire system.
	\item Draw free-body diagram for all objects in a system and solve the forces independently.
	\item We can obtain equations for the forces acting on an objects and set up a system of equations to eliminate variables.
\end{enumerate}
\end{minipage}

\end{document}

