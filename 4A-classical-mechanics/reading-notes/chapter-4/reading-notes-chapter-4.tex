\documentclass{article}
\usepackage{graphicx}
\usepackage{amsmath}
\usepackage{pgfplots}
\usepackage{tikz}
\usepackage{txfonts}
\usepackage{physics}
\usepackage{hyperref}
\usepackage[a4paper, top=1cm, bottom=2cm, left=2cm, right=2cm, includehead, includefoot]{geometry}
\pgfplotsset{compat=1.18}

\begin{document}
\noindent
Physics 4A - Classical Mechanics \hfill Prof. Roger King

\noindent\rule{\textwidth}{0.4pt}

\begin{center}
    \textbf{\LARGE Chapter 4 - Motion in Two and Three Dimensions} \\
    \vspace{12pt}
    \large Aaron W. Tarajos \\
    \textit{\today}
\end{center}

\noindent\rule{\textwidth}{0.4pt}

\section*{4.1 Position and Displacement}
\subsection*{Key Ideas}
\begin{itemize}
    \item \textbf{Position Vector}: The position of a particle in a coordinate system is given by a vector $\vec{r}$, which extends from a reference point (often the origin) to the particle's location. In unit-vector notation:
    \[
    \vec{r} = x\hat{i} + y\hat{j} + z\hat{k}
    \]
    \item \textbf{Displacement}: A particle's displacement $\Delta \vec{r}$ during an interval is the difference between its final and initial position vectors:
    \[
    \Delta \vec{r} = \vec{r}_2 - \vec{r}_1 = \Delta x\hat{i} + \Delta y\hat{j} + \Delta z\hat{k}
    \]
    \item \textbf{Magnitude of Displacement}: The magnitude of the displacement vector can be found using the Pythagorean theorem:
    \[
    |\Delta \vec{r}| = \sqrt{(\Delta x)^2 + (\Delta y)^2 + (\Delta z)^2}
    \]
\end{itemize}
One general way of locating a particle is with a position vector $\vec{r}$, which is a vector that extends from a reference point, usually the origin, to the particle. The position vector in unit-vector notation is given by:
\[
\vec{r} = x\hat{i} + y\hat{j} + z\hat{k}
\]
where $x$, $y$, and $z$ are the scalar components of the vector along the $x$, $y$, and $z$ axes respectively.
As a particle moves, its position vector changes. If the position vector changes from $\vec{r}_1$ to $\vec{r}_2$ during a certain time interval, then the particle's displacement $\Delta \vec{r}$ is given by:
\[
\Delta \vec{r} = \vec{r}_2 - \vec{r}_1
\]
This equation can be expanded in unit-vector notation:
\[
\Delta \vec{r} = (x_2 - x_1)\hat{i} + (y_2 - y_1)\hat{j} + (z_2 - z_1)\hat{k}
\]
The magnitude of the displacement vector, $|\Delta \vec{r}|$, is the distance between the two points and can be calculated using:
\[
|\Delta \vec{r}| = \sqrt{(\Delta x)^2 + (\Delta y)^2 + (\Delta z)^2}
\]
This represents the straight-line distance between the initial and final positions of the particle.

\section*{4.2 Average Velocity and Instantaneous Velocity}
\subsection*{Key Ideas}
\begin{itemize}
    \item \textbf{Average Velocity}: The average velocity of a particle is the ratio of the displacement vector $\Delta \vec{r}$ to the time interval $\Delta t$. It is a vector quantity and is given by:
    \[
    \vec{v}_{avg} = \frac{\Delta \vec{r}}{\Delta t}
    \]
    \item \textbf{Instantaneous Velocity}: The instantaneous velocity is the limit of the average velocity as the time interval approaches zero. It is defined as the derivative of the position vector with respect to time:
    \[
    \vec{v} = \lim_{\Delta t \to 0} \frac{\Delta \vec{r}}{\Delta t} = \frac{d\vec{r}}{dt}
    \]
    \item \textbf{Direction of Velocity}: The instantaneous velocity vector is always tangent to the particle's path at any given point.
\end{itemize}
When a particle moves from one point to another, its average velocity $\vec{v}_{avg}$ can be defined as the displacement $\Delta \vec{r}$ divided by the time interval $\Delta t$:
\[
\vec{v}_{avg} = \frac{\Delta \vec{r}}{\Delta t}
\]
This equation indicates that the direction of $\vec{v}_{avg}$ is the same as the direction of the displacement vector $\Delta \vec{r}$. In component form, this becomes:
\[
\vec{v}_{avg} = \frac{\Delta x}{\Delta t} \hat{i} + \frac{\Delta y}{\Delta t} \hat{j} + \frac{\Delta z}{\Delta t} \hat{k}
\]
For example, if a particle moves through a displacement $\Delta \vec{r} = 12\hat{i} + 3\hat{k}$ meters in a time interval of 2.0 seconds, the average velocity is:
\[
\vec{v}_{avg} = \frac{12\hat{i} + 3\hat{k}}{2.0} = 6.0\hat{i} + 1.5\hat{k} \, \text{m/s}
\]

\subsection*{Instantaneous Velocity}
The instantaneous velocity $\vec{v}$ is the limit of the average velocity as the time interval $\Delta t$ approaches zero:
\[
\vec{v} = \lim_{\Delta t \to 0} \frac{\Delta \vec{r}}{\Delta t} = \frac{d\vec{r}}{dt}
\]
In component form, this is expressed as:
\[
\vec{v} = \frac{dx}{dt} \hat{i} + \frac{dy}{dt} \hat{j} + \frac{dz}{dt} \hat{k}
\]
where $v_x = \frac{dx}{dt}$, $v_y = \frac{dy}{dt}$, and $v_z = \frac{dz}{dt}$ are the scalar components of the velocity vector along the $x$, $y$, and $z$ axes.
The direction of the instantaneous velocity vector is always tangent to the particle’s path. For example, if the position vector $\vec{r}$ of a particle is given as $\vec{r} = x(t)\hat{i} + y(t)\hat{j} + z(t)\hat{k}$, the velocity vector is:
\[
\vec{v} = \frac{d}{dt} (x(t)\hat{i} + y(t)\hat{j} + z(t)\hat{k}) = \frac{dx}{dt} \hat{i} + \frac{dy}{dt} \hat{j} + \frac{dz}{dt} \hat{k}
\]

\section*{Average Acceleration and Instantaneous Acceleration}
\subsection*{Key Ideas}
\begin{itemize}
    \item \textbf{Average Acceleration}: The average acceleration $\vec{a}_{avg}$ is the change in velocity $\Delta \vec{v}$ over a time interval $\Delta t$, expressed as:
    \[
    \vec{a}_{avg} = \frac{\Delta \vec{v}}{\Delta t}
    \]
    \item \textbf{Instantaneous Acceleration}: The instantaneous acceleration $\vec{a}$ is the limit of the average acceleration as the time interval $\Delta t$ approaches zero. It is the derivative of velocity with respect to time:
    \[
    \vec{a} = \frac{d\vec{v}}{dt}
    \]
    \item \textbf{Acceleration Components}: In unit-vector form, acceleration can be written as:
    \[
    \vec{a} = a_x \hat{i} + a_y \hat{j} + a_z \hat{k}
    \]
    where $a_x = \frac{dv_x}{dt}$, $a_y = \frac{dvy}{dt}$, and $a_z = \frac{dv_z}{dt}$.
\end{itemize}
The average acceleration $\vec{a}_{avg}$ of a particle is defined as the ratio of the change in velocity $\Delta \vec{v}$ to the time interval $\Delta t$ during which the change occurs:
\[
\vec{a}_{avg} = \frac{\Delta \vec{v}}{\Delta t}
\]
where $\Delta \vec{v} = \vec{v}_2 - \vec{v}_1$ is the difference between the final velocity $\vec{v}_2$ and the initial velocity $\vec{v}_1$. The direction of $\vec{a}_{avg}$ is the same as the direction of $\Delta \vec{v}$.
In unit-vector notation, the average acceleration can be written as:
\[
\vec{a}_{avg} = \frac{\Delta v_x}{\Delta t} \hat{i} + \frac{\Delta v_y}{\Delta t} \hat{j} + \frac{\Delta v_z}{\Delta t} \hat{k}
\]

\subsection*{Instantaneous Acceleration}
Instantaneous acceleration $\vec{a}$ is the limit of the average acceleration as the time interval $\Delta t$ approaches zero. It is defined as the derivative of the velocity vector with respect to time:
\[
\vec{a} = \lim_{\Delta t \to 0} \frac{\Delta \vec{v}}{\Delta t} = \frac{d\vec{v}}{dt}
\]
In unit-vector notation, the instantaneous acceleration is given by:
\[
\vec{a} = \frac{dv_x}{dt} \hat{i} + \frac{dv_y}{dt} \hat{j} + \frac{dv_z}{dt} \hat{k}
\]
Thus, the scalar components of $\vec{a}$ are:
\[
a_x = \frac{dv_x}{dt}, \quad a_y = \frac{dv_y}{dt}, \quad a_z = \frac{dv_z}{dt}
\]

\subsection*{Acceleration in Two and Three Dimensions}
In two- or three-dimensional motion, the acceleration vector $\vec{a}$ at any instant is tangent to the curve that represents the particle's velocity at that point. If either the magnitude or the direction of the velocity changes, the particle experiences acceleration.

\section*{4.4 Projectile Motion}
\subsection*{Key Ideas}
\begin{itemize}
    \item \textbf{Projectile Motion}: The motion of a particle moving in a vertical plane under the influence of gravity, with no horizontal acceleration and constant vertical acceleration due to gravity.
    \item \textbf{Components of Motion}: The horizontal and vertical components of motion are independent of each other.
    \item \textbf{Equations of Motion}:
    \[
    x = x_0 + (v_0 \cos \theta)t
    \]
    \[
    y = y_0 + (v_0 \sin \theta)t - \frac{1}{2} g t^2
    \]
    \[
    v_y = v_0 \sin \theta - g t
    \]
    \item \textbf{Path of the Projectile}: The path followed is parabolic and is given by the trajectory equation:
    \[
    y = ( \tan \theta) x - \frac{g x^2}{2 (v_0 \cos \theta)^2}
    \]
\end{itemize}
Projectile motion occurs when an object is launched into the air with some initial velocity and its motion is subject to the gravitational force. A projectile might be a tennis ball, a baseball, or any object that moves under the influence of gravity, with no other forces acting on it, such as air resistance. The key feature of projectile motion is that the horizontal and vertical components of the motion are independent.

\subsection*{Components of Motion}

In projectile motion, the initial velocity $\vec{v}_0$ can be resolved into horizontal and vertical components:
\[
v_0x = v_0 \cos \theta, \quad v_0y = v_0 \sin \theta
\]
The horizontal motion has no acceleration, meaning the velocity in the horizontal direction remains constant:
\[
x = x_0 + v_0x t = x_0 + (v_0 \cos \theta)t
\]
In the vertical direction, the motion is influenced by gravity, with acceleration $a = -g$. The position in the vertical direction at time $t$ is:
\[
y = y_0 + v_0y t - \frac{1}{2} g t^2 = y_0 + (v_0 \sin \theta)t - \frac{1}{2} g t^2
\]
The vertical velocity decreases as the projectile moves upward, reaches zero at the maximum height, and then increases in the downward direction:
\[
v_y = v_0 \sin \theta - g t
\]

\subsection*{Trajectory of a Projectile}

The trajectory of a projectile is parabolic in nature. By eliminating time $t$ from the equations of horizontal and vertical motion, we can derive the equation for the path:
\[
y = ( \tan \theta) x - \frac{g x^2}{2 (v_0 \cos \theta)^2}
\]
This equation shows that the path of the projectile is a parabola.

\subsection*{Range of the Projectile}

The range $R$ of the projectile, which is the horizontal distance the projectile travels before returning to its initial height, is given by:
\[
R = \frac{v_0^2 \sin 2\theta}{g}
\]
This equation assumes that the projectile returns to the same height from which it was launched.

\section*{4.5 Uniform Circular Motion}
\subsection*{Key Ideas}
\begin{itemize}
    \item \textbf{Uniform Circular Motion}: When a particle moves in a circle at constant speed, it is in uniform circular motion.
    \item \textbf{Centripetal Acceleration}: Even though the speed is constant, the velocity is not because its direction changes. The particle experiences an acceleration called centripetal acceleration, directed toward the center of the circle:
    \[
    a = \frac{v^2}{r}
    \]
    \item \textbf{Period of Revolution}: The time $T$ it takes for the particle to make one complete revolution around the circle is related to the radius $r$ and speed $v$:
    \[
    T = \frac{2\pi r}{v}
    \]
\end{itemize}
A particle is in uniform circular motion if it travels around a circle or a circular arc at a constant speed. Even though the speed does not change, the velocity vector changes direction at every point on the circle, meaning the particle is accelerating.

\subsection*{Centripetal Acceleration}

The acceleration responsible for the change in direction of the velocity is directed radially inward toward the center of the circle and is called centripetal acceleration. The magnitude of the centripetal acceleration is:
\[
a = \frac{v^2}{r}
\]
where $v$ is the speed of the particle and $r$ is the radius of the circle. Centripetal acceleration does not change the magnitude of the velocity but continuously changes its direction.

\subsection*{Velocity and Acceleration Vectors}

At every point in the circular path, the velocity vector is tangent to the circle, while the acceleration vector points toward the center. These vectors are perpendicular to each other throughout the motion.

\subsection*{Period of Revolution}

The period $T$ of a particle in uniform circular motion is the time it takes for the particle to make one complete revolution around the circle. It can be calculated using the relationship:
\[
T = \frac{2\pi r}{v}
\]
where $r$ is the radius of the circle and $v$ is the speed. The particle travels the circumference of the circle ($2\pi r$) in this time.

\section*{4.6 Relative Motion in One Dimension}
\subsection*{Key Ideas}
\begin{itemize}
    \item \textbf{Relative Motion}: The velocity of an object can be different when measured from different reference frames, depending on the motion of the observer.
    \item \textbf{Reference Frames}: A reference frame is the object or system from which motion is observed and measured.
    \item \textbf{Relative Velocity}: The velocity of object P as measured from frame A is related to the velocity of object P as measured from frame B by the relative velocity of frames A and B:
    \[
    v_{PA} = v_{PB} + v_{BA}
    \]
    \item \textbf{Constant Velocity}: In this section, we only consider reference frames that move at constant velocities relative to each other.
\end{itemize}
Relative motion occurs when two reference frames are moving relative to each other at constant velocity along a single axis. At a given instant, the positions of the car relative to both observers can be related by:
\[
x_{PA} = x_{PB} + x_{BA}
\]
The derivative of this equation with respect to time gives us the relationship between the velocities in the two frames:
\[
v_{PA} = v_{PB} + v_{BA}
\]
Since $v_{BA}$ is constant, the accelerations of the car relative to both observers are the same. That is:
\[
a_{PA} = a_{PB}
\]

\section*{4.7 Relative Motion in Two Dimensions}
\section*{Key Ideas}
\begin{itemize}
    \item \textbf{Relative Motion in Two Dimensions}: When observing motion from different reference frames moving relative to each other, the velocity measured in one frame is different from the velocity measured in the other frame.
    \item \textbf{Vector Addition}: The velocity of an object in one frame of reference is the sum of its velocity relative to a second reference frame and the velocity of the second reference frame relative to the first:
    \[
    \vec{v}_{PA} = \vec{v}_{PB} + \vec{v}_{BA}
    \]
    \item \textbf{Acceleration}: Observers in different reference frames moving at constant velocity relative to each other will measure the same acceleration for a particle:
    \[
    \vec{a}_{PA} = \vec{a}_{PB}
    \]
\end{itemize}
When observing motion in two dimensions, the velocity of an object as seen from one reference frame can be different from the velocity as seen from another reference frame moving relative to the first.\vspace{12pt}\\
Consider two observers, one at the origin of reference frame A and the other at the origin of reference frame B. Frame B moves with a constant velocity $\vec{v}_{BA}$ relative to frame A. At a given instant, the position vector of the object P as measured by the observers is given by:
\[
\vec{r}_{PA} = \vec{r}_{PB} + \vec{r}_{BA}
\]
Taking the time derivative of this equation gives the velocities:
\[
\vec{v}_{PA} = \vec{v}_{PB} + \vec{v}_{BA}
\]
Here, $\vec{v}_{PA}$ is the velocity of particle P relative to frame A, $\vec{v}_{PB}$ is the velocity of P relative to frame B, and $\vec{v}_{BA}$ is the velocity of frame B relative to frame A.\vspace{12pt}\\
Since the relative velocity $\vec{v}_{BA}$ is constant, the time derivative of the relative velocity is zero. This means that the acceleration of the particle relative to both frames is the same:
\[
\vec{a}_{PA} = \vec{a}_{PB}
\]
Thus, even though the velocities measured by the two observers are different, the acceleration measured by both is the same.


\end{document}


